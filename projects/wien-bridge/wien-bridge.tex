\documentclass{article}
\usepackage[letterpaper, margin=0.9in]{geometry}
\usepackage{graphicx}
\usepackage{amsmath}
\usepackage{comment} % enables the use of multi-line comments (\ifx \fi) 
\usepackage{lipsum} %This package just generates Lorem Ipsum filler text. 
%\usepackage{fullpage} % changes the margin

\begin{document}

K is the gain of the amplifier and AB is the loop gain. 	Let $\tau$ = CR.
\begin {gather*}
	A = K = \frac{R_1 + R_2}{R_2}\\
	B = \frac{-V_+}{V_{out}} = \frac{Z_2}{Z_1 + Z_2}\\
	Z_1 = \frac{1}{sC} + R = \frac{1 + sCR}{sC}\\
	Z_2 = \frac{1}{sC}||R = \frac{R}{1+sCR}\\
	B = -\frac{R}{R+\frac{{(1+sCR)}^2}{sC}} = -\frac{sCR}{1+3sCR+s^2C^2R^2}\\
	AB = -\frac{-sK\tau}{1+3s\tau+s^2{\tau}^2}\\
\end {gather*}
This transfer function has a zero at $\omega$ = 0 and two poles at ${\omega}_o = \frac{1}{\tau}$. |AB(${\omega}_o$)| = $\frac{K}{3}$. A Nyquist plot is a plot of the frequency reponse of a system. The stability of an LTI closed-loop system is done by examining the Nyquist plot of the open loop system. If the point -1 + j0 is encircled by the Nyquist plot, then the closed-loop system in unstable. If the plot does not encircle the -1+j0 point, then it is stable. For an oscillator, the poles are ideally on the imaginary axis. This is expressed in the Nyquist plot by the loop passing through the -1+j0 point. For the Wien-Bridge oscillator, the Nyquist plot passes through the -1+j0 point when K = 3. This results in a closed loop gain of:
\begin {gather*}
	H(s) = \frac{V_{out}}{V_{in}} = \frac{A}{1 + AB}\\
	= \frac{3}{1-\frac{3s\tau}{1+3s\tau+s^2{\tau}^2}} = \frac{3(1+3s\tau+s^2{\tau}^2)}{1+s^2{\tau}^2}\\
\end {gather*}
with poles at $\frac{{\pm}j}{\tau}$. This will produce an output sine wave that oscillates from rail to rai at a frequency of $f = \frac{1}{2{\pi}RC}$. There are a variety of methods that can be used to limit the output of the oscillator, however many of them will produce additional harmonics. The Wien-Bridge oscillator was the first product made by Hewlitt-Packard. Their original design used a lightbulb in place of $R_2$ which provided a soft rise to a limited output, reducing the amount of harmonics.
\end{document}